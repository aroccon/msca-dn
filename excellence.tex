\section{Excellence}
\note{Should start on page 5; aim for 10 to 12 pages.}

\subsection{Quality and pertinence of the project's research and innovation objectives}
\note{and the extent to which they are ambitious, and go beyond the state of the art}

\subsubsection{Introduction, objectives and overview of the research programme}
\note{it should be explained how the individual projects of the recruited researchers will be integrated into --- and contribute to --- the overall research programme. All proposals should describe the research projects in the context of a doctoral training programme.  Are the objectives measurable and verifiable? Are they realistically achievable?}


\itncaption{Table 1.1: Work Package (WP) List}
\note{The WP names are defined in common.tex}
\begin{itntable}{|>{\ra}p{10mm}|>{\ra}p{35mm}|>{\ra}p{15mm}|>{\ra}p{10mm}|>{\ra}p{10mm}|>{\ra}p{20mm}|>{\ra}p{20mm}|>{\ra}p{25mm}|}
    \colorrow
    \hline
    \textbf{WP No.} &
    \textbf{WP Title} &
    \textbf{Lead Beneficiary No.} &
    \textbf{Start Month} &
    \textbf{End Month} &
    \textbf{Activity Type} &
    \textbf{Lead Beneficiary Short Name} &
    \textbf{ESR Involvement} \\
    \hline
    &&&&&&& \\
    \hline
\end{itntable}

\newpage
\subsection{Soundness of the proposed methodology}
\note{including interdisciplinary approaches, consideration of the gender dimension and other diversity aspects if relevant for the research project, and the quality and appropriateness of open science practices}

\subsubsection{Research data management and management of other research outputs}

% Should be 1.3 as of 2021
\subsection{Quality and credibility of the training programme}
\note{including transferable skills, inter/multi-disciplinary, inter-sectoral and, where appropriate, gender aspects}

\subsubsection{Overview and content structure of the doctoral training programme}

\itncaption{Table 1.3a: Recruitment Deliverables per Beneficiary}
\begin{itntable}{|>{\ra}p{30mm}|>{\ra}p{30mm}|>{\ra}p{30mm}|>{\ra}p{30mm}|>{\ra}p{30mm}|}
    \colorrow
    \hline
    \textbf{Researcher No.} &
    \textbf{Recruiting Participant (short name)} &
    \textbf{PhD awarding entities} &
    \textbf{Planned Start Month 0--45} &
    \textbf{Duration (months) 3--36} \\
    \hline
    &&&& \\
    \hline
    Total & & & & XXX \\
    \hline
    \end{itntable}

\itncaption{Table 1.3b: Main Network-Wide Training Events, Conferences and Contribution of Beneficiaries}
\begin{itntable}{|>{\ra}p{10mm}|>{\ra}p{85mm}|>{\ra}p{20mm}|>{\ra}p{20mm}|>{\ra}p{20mm}|}
    \colorrow
    \hline &
    \textbf{Main Training Events \& Conferences} &
    \textbf{ECTS (if any)} &
    \textbf{Lead Institution} &
    \textbf{Action Month (estimated)} \\
    \hline
    &&&& \\
    \hline
\end{itntable}


\subsubsection{Qualifications and supervision experience of supervisors}


\subsubsection{Quality of the joint supervision arrangements}
\note{mandatory for EID and EJD}
\note{Would make sense here to emphasise the multidisciplinarity - add a table?}

\subsubsection{Contribution of all participating organisations to the research and training programme}


\subsubsection{Synergies between participating organisations}

\subsubsection{Exposure of recruited researchers to different (research)
  environments, and the complementarity thereof}

%%% Local Variables:
%%% mode: latex
%%% TeX-master: "master"
%%% TeX-PDF-mode: t
%%% End:
