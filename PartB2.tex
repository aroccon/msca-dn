\documentclass[11pt]{article}
\usepackage[utf8]{inputenc}

% Generic DN style
\usepackage[ACRONYM]{msca-dn}

% Common between all documents
%%
%% Common definitions across Part B1, Part B2 and hence master.tex
%%

\usepackage{graphicx}

\usepackage{textcomp} % for \texteuro
\usepackage{pgfgantt}
\usepackage{wrapfig}

% Citations
\addbibresource{itn-refs.bib}

% The total number of ESRs
\newcommand{\researchercount}{ten\xspace}

%%% Local Variables:
%%% mode: latex
%%% TeX-master: "master"
%%% TeX-PDF-mode: t
%%% End:


\begin{document}

% These depend on PartB1
\setcounter{section}{3}
\setcounter{page}{34}

% \mscpage{START PAGE}
\tableofcontents



\newpage

\section{Recruitment strategy (including how the project will strive to adhere to the Code of Conduct for the recruitment of researcher.}

 The following sections of the European Code of Conduct for the Recruitment of Researchers refer specifically to recruitment and selection:

Recruitment\\
Employers and/or funders should establish recruitment procedures which are open, efficient, transparent, supportive and internationally comparable, as well as tailored to the type of positions advertised.
Advertisements should give a broad description of knowledge and competencies required, and should not be so specialised as to discourage suitable applicants. Employers should include a description of the working conditions and entitlements, including career development prospects. Moreover, the time allowed between the advertisement of the vacancy or the call for applications and the deadline for reply should be realistic.

Selection\\
Selection committees should bring together diverse expertise and competences and should have an adequate gender balance and, where appropriate and feasible, include members from different sectors (academic and non-academic) and disciplines, including from other countries and with relevant experience to assess the candidate. Whenever possible, a wide range of selection practices should be used, such as external expert assessment and face-to-face interviews. Members of selection panels should be adequately trained.






\section{Network organization}

Please explain the composition and organisation of the Supervisory board, and how it will strive to adhere to the MSCA guidelines on supervision.
Please insert a table that displays the names and gender of supervisors for all Doctoral Candidates, to be adapted to your particular proposal;







\section{Supervisory board}

Please explain the composition and organisation of the Supervisory board, and how it will strive to adhere to the Marie Skłodowska-Curie actions guidelines on supervision.
Please insert a table that displays the names and gender of supervisors for all Doctoral Candidates, to be adapted to your particular proposal;






\begin{msctable}{|>{\ra}p{40mm}|>{\ra}p{40mm}|>{\ra}p{20mm}|>{\ra}p{40mm}|>{\ra}p{20mm}|}
    \hline
    \colorrow
    \textbf{Doctoral Candidate} &
    \textbf{Main Supervisor} &
    \textbf{Gender} &
    \textbf{Co-supervisor} &
    \textbf{Gender} \\
    \hline 
    DC1 & & & & \\
    \hline
    DC2 & & & & \\
    \hline
    DC3 & & & & \\
    \hline
    DC4 & & & & \\
    \hline
\end{msctable}


\section{Environmental aspects in light of the MSCA Green Charter}

Please explain how the proposed project would strive to adhere to the MSCA Green Charter during its implementation.






\section{Participating organizations}


All organisations (whether beneficiaries or associated partners) must complete the appropriate table below. Complete one table of maximum one page per beneficiary and half a page per associated partner (minimum font size: 9). Associated partners linked to a beneficiary should be described separately.

\newpage

For \textbf{beneficiaries}:
 
\begin{msctable}{|>{\ra}p{85mm}|>{\ra}p{85mm}|}
    \hline
    \colorrow
    \textbf{Beneficiary Legal Name:} & \\
    \hline 
    General Description &
    Short description of the activities relevant to the action \\
    \hline
    Role and Commitment of key persons (including supervisors)  &
    Including names, title and the intended extent of involvement in the action (in percentage of full-time employment) of the key scientific staff who will be involved in the research, training and supervision \\
    \hline
    Key Research Facilities, Infrastructure and Equipment & 
    Outline the key facilities and infrastructure available and demonstrate that each team has sufficient capacity to host and/or offer a suitable environment for supervising the research and training of the recruited researchers\\
    \hline
    Status of Research Premises & 
    Please explain the status of the beneficiary's research facilities – i.e. are they owned by the beneficiary or rented by it? Are its research premises wholly independent from other beneficiaries and/or associated partners in the consortium?    \\
    \hline
    Previous Involvement in Research and Training Programmes, including H2020 ITN &
    Detail any relevant EU, national or international research and training actions/projects in which the beneficiary has previously participated. Please clearly mention any previous involvement in H2020 ITN funded project(s), including project(s) acronym and reference number.\\
    \hline 
    Current Involvement in Research and Training Programmes, including H2020 ITN & 
    Detail any relevant EU, national or international research and training actions/projects in which the beneficiary is currently participating. Please clearly mention any current involvement in ongoing ITN funded project(s), including project(s) acronym and reference number. \\ 
    \hline
    Relevant Publications/datasets/ softwares/ Innovation Products/ other achievements&
    Max. 5
Key elements of the achievement, including a short qualitative assessment of its impact and (where available) its digital object identifier (DOI) or other type of persistent identifier (PID).
Publications, in particular journal articles, are expected to be open access. Datasets are expected to be FAIR and ‘as open as possible, as closed as necessary’.\\
    \hline
\end{msctable}

    
For \textbf{associated partners}:

    
\begin{msctable}{|>{\ra}p{85mm}|>{\ra}p{85mm}|}
    \hline
    \colorrow
    \textbf{Associated Partner Legal Name: } & \\
    \hline 
    General Description &  \\
    \hline
    Key Persons and Expertise
 & \\
     \hline
    Key Research Facilities, Infrastructure and Equipment & 
   \\
    \hline
    Previous and Current Involvement in Research and Training Programmes, including H2020 ITN & \\
    \hline 
    Relevant Publications/datasets/ softwares/ Innovation Products/ other achievements&
    Max. 3, Key elements of the achievement, including a short qualitative assessment of its impact and (where available) its digital object identifier (DOI) or other type of persistent identifier (PID).
Publications, in particular journal articles, are expected to be open access. Datasets are expected to be FAIR and ‘as open as possible, as closed as necessary’..\\
    \hline
\end{msctable}
    
    
For \textbf{associated partner linked to a beneficiary:}

    
\begin{msctable}{|>{\ra}p{85mm}|>{\ra}p{85mm}|}
    \hline
    \colorrow
    \textbf{Associated Partner linked to a beneficiary Legal Name: } & \\
    \hline 
    General Description &\\
        \hline
    Key Persons and Expertise
 & \\
     \hline
    Key Research Facilities, Infrastructure and Equipment & 
   \\
    \hline
    Previous and Current Involvement in Research and Training Programmes, including H2020 ITN & \\
    \hline 
    Relevant Publications/datasets/ softwares/ Innovation Products/ other achievements&
    Max. 3, Key elements of the achievement, including a short qualitative assessment of its impact and (where available) its digital object identifier (DOI) or other type of persistent identifier (PID).
Publications, in particular journal articles, are expected to be open access. Datasets are expected to be FAIR and ‘as open as possible, as closed as necessary’..\\
    \hline
\end{msctable}




\section{Letters of  pre-agreement (for DN-JD)}

For DN-JD, letters of pre-agreement must also be included from those academic beneficiaries/associated partners that will award the doctoral degrees, in part B (document 2) of the proposal. These letters should be signed by an authorised legal representative of the organisation in question so as to offer reasonable assurance regarding the commitment to award the joint, double or multiple doctoral degree(s). These letters should also indicate agreement with the principle that the awarding of such degrees is a precondition for funding. A template for these letters is provided below and must be followed by all academic DN-JD applicants awarding the doctoral degree(s). 
In case the letter does not follow in full the template or fails to give enough assurance on the commitment in the project (e.g. no signature, wrong proposal references, outdated letter…), the experts may penalise the proposal on these aspects under the implementation evaluation criterion. Missing letters of pre-agreement will lead to the exclusion of the entity, which may affect the eligibility of the proposal.
Letters of pre-agreement must be included in the PDF file (Part B, document 2); these should not be attached in a separate PDF file or as an embedded file since this makes them invisible.






\newpage

\mscpage{END PAGE}

\end{document}


%%% Local Variables:
%%% mode: latex
%%% TeX-master: t
%%% TeX-PDF-mode: t
%%% End:
