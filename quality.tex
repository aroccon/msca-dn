\section{Quality and Efficiency of the Implementation}
\note{\textbf{Quality and efficiency of the implementation – aspects to be taken into account}
\begin{compactitem}
  \item Quality and effectiveness of the work plan, assessment of risks and appropriateness of the effort assigned to work packages.
  \item Quality, capacity and role of each participant, including hosting arrangements and extent to which the consortium as a whole brings together the necessary expertise.
\end{compactitem}}

\subsection{Quality and effectiveness of the work plan, assessment of risks and appropriateness of the effort assigned to work packages}

\subsubsection{Work Packages (WP) list}
\note{please include table 3.1a}

\msccaption{Table 3.1a: Work Package (WP) List}
\note{The WP names are defined in common.tex}
\begin{msctable}{|>{\ra}p{10mm}|>{\ra}p{35mm}|>{\ra}p{15mm}|>{\ra}p{10mm}|>{\ra}p{10mm}|>{\ra}p{20mm}|>{\ra}p{20mm}|>{\ra}p{25mm}|}
    \colorrow
    \hline
    \textbf{WP No.} &
    \textbf{WP Title} &
    \textbf{Lead Beneficiary No.} &
    \textbf{Lead Beneficiary Short Name} &
    \textbf{Start Month} &
    \textbf{End Month} &
    \textbf{Activity Type} &
    \textbf{Researcher Involvement} \\
    \hline
    &&&&&&& \\
    \hline
\end{msctable}

\msccaption{Table 3.1b: Description of Work Packages}

\begin{mscwp}{MX -- MY}{Title}{Lead}
    \mscwppar{Objectives}{%
    }\\
    \hline    
    \mscwppar{Description of Work and Role of Specific Beneficiaries / Associated partners}{%
    }\\
    \mscwptask{Task name}{Task description}\\
    \hline
    \mscwppar{Description of Deliverables}{%
    }\\
\end{mscwp}


\subsubsection{Deliverables list}
\note{please include table 3.1c, including the awarding of doctoral degrees}

\note{%
    Type of deliverable:
    \begin{compactdesc}
    \item[R] Report
    \item[ADM] Administrative (website completion, recruitment completion, etc.)
    \item[PDE] dissemination and/or exploitation of results
    \item[OTHER] Other, including coordination.
    \end{compactdesc}
    Dissemination level:
    \begin{compactdesc}
    \item[PU] Public: fully open, e.g. web
    \item[CO] Confidential: restricted to consortium, other designated entities (as appropriate) and Commission services; Please consider that deliverables marked as ``PU'' will automatically be published on CORDIS once approved: the applicants should therefore consider the relevance of marking a deliverable as ``PU''
    \item[CI] Classified: classified information as intended in Commission Decision 2001/844/EC
    \end{compactdesc}
}
\msccaption{Table 3.1c: Deliverables List}
\begin{msctable}{|>{\ra}p{15mm}|>{\ra}p{50mm}|>{\ra}p{15mm}|>{\ra}p{20mm}|>{\ra}p{15mm}|>{\ra}p{15mm}|>{\ra}p{15mm}|}
    \hline
    \colorrow
    \multicolumn{7}{|l|}{\textbf{Scientific Deliverables}} \\
    \hline
    \colorrow
    \textbf{Deliverable Number} &
    \textbf{Deliverable Title} &
    \textbf{WP No.} &
    \textbf{Lead Beneficiary Short Name} &
    \textbf{Type} &
    \textbf{Dissem. Level} &
    \textbf{Due Date} \\
    \hline
    % Split it here if need be
\end{msctable}
\begin{msctable}{|>{\ra}p{15mm}|>{\ra}p{50mm}|>{\ra}p{15mm}|>{\ra}p{20mm}|>{\ra}p{15mm}|>{\ra}p{15mm}|>{\ra}p{15mm}|}
    \hline
    \colorrow
    \multicolumn{7}{|l|}{\textbf{Management, Training, Recruitment and Dissemination Deliverables}} \\
    \hline
    \colorrow
    \textbf{Deliverable Number} &
    \textbf{Deliverable Title} &
    \textbf{WP No.} &
    \textbf{Lead Beneficiary Short Name} &
    \textbf{Type} &
    \textbf{Dissem. Level} &
    \textbf{Due Date} \\
    \hline
\end{msctable}

\subsubsection{Milestones list}

\msccaption{Table 3.1d: Milestones List}

\begin{msctable}{|>{\ra}p{15mm}|>{\ra}p{50mm}|>{\ra}p{15mm}|>{\ra}p{20mm}|>{\ra}p{15mm}|>{\ra}p{35mm}|}
    \hline
    \colorrow
    \textbf{Number} &
    \textbf{Title} &
    \textbf{Related Work Packages} &
    \textbf{Lead Beneficiary} &
    \textbf{Due Date} &
    \textbf{Means of Verification} \\
    \hline
\end{msctable}

\subsubsection{Recruitment table}

\msccaption{Table 3.1e: Recruitment Table per Beneficiary}
\begin{msctable}{|>{\ra}p{30mm}|>{\ra}p{30mm}|>{\ra}p{30mm}|>{\ra}p{30mm}|>{\ra}p{30mm}|}
    \colorrow
    \hline
    \textbf{Researcher No.} &
    \textbf{Recruiting Participant (short name)} &
    \textbf{PhD awarding entities} &
    \textbf{Planned Start Month 0--45} &
    \textbf{Duration (months) 3--36} \\
    \hline
    &&&& \\
    \hline
    Total & & & & XXX \\
    \hline
\end{msctable}

\subsubsection{Individual projects, including secondment plan}

\msccaption{Table 3.1f: Individual Research Projects}

\note{%
\begin{mscrp}
    X & X & X & X & X & X \\
    \hline
    \mscrppar{Project Title and Work Package(s) to which it is related}{
    }\\ \hline
    \mscrppar{Objectives}{%
    }\\ \hline
    \mscrppar{Expected results}{%
    }\\ \hline
    \mscrppar{Planned secondment(s)}{%
    }\\
\end{mscrp}
}

\subsubsection{Implementation risks}
\msccaption{Table 3.1e: Implementation Risks}
\begin{msctable}{|>{\ra}p{60mm}|>{\ra}p{15mm}|>{\ra}p{90mm}|}
    \hline
    \colorrow
    \textbf{Description of risk} (likelihood / severity) &
    \textbf{WPs involved} &
    \textbf{Proposed risk-mitigation measures} \\
    \hline
\end{msctable}

\subsubsection{Supervisory board}
\note{including gender aspects in the decision making of the board}

\subsubsection{Recruitment strategy}
\note{including gender aspects in the selection process}

% For DN-JD, joint admission, selection, supervision, monitoring and assessment procedures (if not applicable, please remove).

\note{\textbf{Recruitment}
Employers and/or funders should establish recruitment procedures which are open, efficient, transparent, supportive and internationally comparable, as well as tailored to the type of positions advertised. Advertisements should give a broad description of knowledge and competencies required, and should not be so specialised as to discourage suitable applicants. Employers should include a description of the working conditions and entitlements, including career development prospects. Moreover, the time allowed between the advertisement of the vacancy or the call for applications and the deadline for reply should be realistic.\\
\textbf{Selection}
Selection committees should bring together diverse expertise and competences and should have an adequate gender balance and, where appropriate and feasible, include members from different sectors
(academic and non-academic) and disciplines, including from other countries and with relevant experience to assess the candidate. Whenever possible, a wide range of selection practices should be used, such as external expert assessment and face-to-face interviews. Members of selection panels should be adequately trained.}

\subsection{Quality, capacity and role of each participant, including hosting arrangements and extent to which the consortium as a whole brings together the necessary expertise}

\subsubsection{Appropriateness of the infrastructure and capacity of each participating organisation}
\note{as outlined in Section 6 (Participating Organisations), in light of the tasks allocated to them in the action}

\subsubsection{Consortium composition and exploitation of participating organisations' complementarities}
\note{explain the compatibility and coherence between the tasks attributed to each beneficiary/associated partner in the action, including in light of their experience; Show how this includes expertise in social sciences and humanities, open science practices, and gender aspects of R\&I, as appropriate.}

\subsubsection{Commitment of beneficiaries and associated partners to the programme}
\note{(for associated partners, please see also sections 6 and 7). The role of associated partners and their active contribution to the research and training activities should be described. A letter of commitment shall also be provided in section 7 and must follow the template (included within the PDF file, but outside the page limit).}

\subsubsection{Funding of non-associated third countries (if applicable)}
\note{Only entities from EU Member States, from Horizon Europe Associated Countries or from countries listed in the HE Programme guide are automatically eligible for EU funding. If one or more of the beneficiaries requesting EU funding is based in a country that is not automatically eligible for such funding, the application shall explain in terms of the objectives of the action why such funding would be essential. Only
in exceptional cases will these organisations receive EU funding. The same applies for international organisations other than IERO.}


%%% Local Variables:
%%% mode: latex
%%% TeX-master: "master"
%%% TeX-PDF-mode: t
%%% End:
