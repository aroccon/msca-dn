\section{Quality and Efficiency of the Implementation}

\subsection{Quality and effectiveness of the work plan, assessment of risks and appropriateness of the effort assigned to work packages}
\note{including appropriateness of the allocation of tasks and resources (including awarding of the doctoral degrees for EID and EJD)}

\subsubsection{Work Packages description}
\note{please include table 3.1a}

\msccaption{Table 3.1a: Work Package (WP) List}
\note{The WP names are defined in common.tex}
\begin{msctable}{|>{\ra}p{10mm}|>{\ra}p{35mm}|>{\ra}p{15mm}|>{\ra}p{10mm}|>{\ra}p{10mm}|>{\ra}p{20mm}|>{\ra}p{20mm}|>{\ra}p{25mm}|}
    \colorrow
    \hline
    \textbf{WP No.} &
    \textbf{WP Title} &
    \textbf{Lead Beneficiary No.} &
    \textbf{Start Month} &
    \textbf{End Month} &
    \textbf{Activity Type} &
    \textbf{Lead Beneficiary Short Name} &
    \textbf{Researcher Involvement} \\
    \hline
    &&&&&&& \\
    \hline
\end{msctable}

\msccaption{Table 3.1b: Description of Work Packages}

% Generic template
\note{%}
\begin{mscwp}{MX -- MY}{Title}{Lead}
    \mscwppar{Objectives}{%
    }\\
    \hline    
    \mscwppar{Description of Work and Role of Specific Beneficiaries / Associated partners}{%
    }\\
    \mscwptask{Task name}{Task description}\\
    \hline
    \mscwppar{Description of Deliverables}{%
    }\\
\end{mscwp}
}


\subsubsection{List of major deliverables}
\note{please include table 3.1b, including the awarding of doctoral degrees, where applicable}

\note{%
    Type of deliverable:
    \begin{compactdesc}
    \item[R] Report
    \item[ADM] Administrative (website completion, recruitment completion, etc.)
    \item[PDE] dissemination and/or exploitation of results
    \item[OTHER] Other, including coordination.
    \end{compactdesc}
    Dissemination level:
    \begin{compactdesc}
    \item[PU] Public: fully open, e.g. web
    \item[CO] Confidential: restricted to consortium, other designated entities (as appropriate) and Commission services; Please consider that deliverables marked as ``PU'' will automatically be published on CORDIS once approved: the applicants should therefore consider the relevance of marking a deliverable as ``PU''
    \item[CI] Classified: classified information as intended in Commission Decision 2001/844/EC
    \end{compactdesc}
}
\msccaption{Table 3.1b: Deliverables List}
\begin{msctable}{|>{\ra}p{15mm}|>{\ra}p{50mm}|>{\ra}p{15mm}|>{\ra}p{20mm}|>{\ra}p{15mm}|>{\ra}p{15mm}|>{\ra}p{15mm}|}
    \hline
    \colorrow
    \multicolumn{7}{|l|}{\textbf{Scientific Deliverables}} \\
    \hline
    \colorrow
    \textbf{Deliverable Number} &
    \textbf{Deliverable Title} &
    \textbf{WP No.} &
    \textbf{Lead Beneficiary Short Name} &
    \textbf{Type} &
    \textbf{Dissem. Level} &
    \textbf{Due Date} \\
    \hline
    % Split it here if need be
\end{msctable}
\begin{msctable}{|>{\ra}p{15mm}|>{\ra}p{50mm}|>{\ra}p{15mm}|>{\ra}p{20mm}|>{\ra}p{15mm}|>{\ra}p{15mm}|>{\ra}p{15mm}|}
    \hline
    \colorrow
    \multicolumn{7}{|l|}{\textbf{Management, Training, Recruitment and Dissemination Deliverables}} \\
    \hline
    \colorrow
    \textbf{Deliverable Number} &
    \textbf{Deliverable Title} &
    \textbf{WP No.} &
    \textbf{Lead Beneficiary Short Name} &
    \textbf{Type} &
    \textbf{Dissem. Level} &
    \textbf{Due Date} \\
    \hline
\end{msctable}

\subsubsection{List of major milestones}

\msccaption{Table 3.1c: Milestones List}

\begin{msctable}{|>{\ra}p{15mm}|>{\ra}p{50mm}|>{\ra}p{15mm}|>{\ra}p{20mm}|>{\ra}p{15mm}|>{\ra}p{35mm}|}
    \hline
    \colorrow
    \textbf{Number} &
    \textbf{Title} &
    \textbf{Related Work Packages} &
    \textbf{Lead Beneficiary} &
    \textbf{Due Date} &
    \textbf{Means of Verification} \\
    \hline
\end{msctable}

\subsubsection{Fellow's individual projects, including secondment plan}

\msccaption{Table 3.1d: Individual Research Projects}

\note{%
\begin{mscrp}
    X & X & X & X & X & X \\
    \hline
    \mscrppar{Project Title and Work Package(s) to which it is related}{
    }\\ \hline
    \mscrppar{Objectives}{%
    }\\ \hline
    \mscrppar{Expected results}{%
    }\\ \hline
    \mscrppar{Planned secondment(s)}{%
    }\\
\end{mscrp}
}


\subsubsection{Network organisation}
\note{including financial management strategy, strategy for dealing with
  scientific misconduct}


\subsubsection{Risk management}
\note{at consortium level (including table 3.2a)}

\msccaption{Table 3.1e: Implementation Risks}
\begin{msctable}{|>{\ra}p{60mm}|>{\ra}p{15mm}|>{\ra}p{90mm}|}
    \hline
    \colorrow
    \textbf{Description of risk} (likelihood / severity) &
    \textbf{WPs involved} &
    \textbf{Proposed risk-mitigation measures} \\
    \hline
\end{msctable}

\subsubsection{Intellectual Property Rights (IPR)}


\subsubsection{Gender aspects}
\label{sec:gender}
\note{both at the level of recruitment and that of decision-making within the action}


\subsubsection{Environmental aspects in light of the MSCA Green Charter}


\subsection{Quality, capacity and role of each participant, including hosting arrangements and extent to which the consortium as a whole brings together the necessary expertise}

\subsubsection{Appropriateness of the infrastructure and capacity of each participating
  organisation}
\note{Explain the appropriateness of the infrastructure of each participating
  organisation, as outlined in Section 4 (Participating Organisations), in
  light of the tasks allocated to them in the action}


\subsubsection{Consortium composition and exploitation of participating
  organisations' complementarities}
\note{explain the compatibility and coherence between the tasks attributed to each beneficiary/partner organisation in the action, including in light of their experience}
\label{sec:composition}


\subsubsection{Commitment of beneficiaries and associated partners to the programme}
\note{for partner organisations, please see also sections 5 and 7) i) Funding of non-associated third countries (if applicable): Only entities from EU Member States, from Horizon 2020 Associated Countries or from countries listed in General Annex A to the Work Programme are automatically eligible for EU funding. If one or more of the beneficiaries requesting EU funding is based in a country that is not automatically eligible for such funding, the application shall explain in terms of the objectives of the action why such funding would be essential. Only in exceptional cases will these organisations receive EU funding.49 The same applies for international organisations other than IEIO.  ii) Partner organisations: The role of partner organisations and their active contribution to the research and training activities should be described. A letter of commitment shall also be provided in section 7 and must follow the template (included within the PDF file, but outside the page limit)}


\subsubsection{Funding of non-associated third countries}



%%% Local Variables:
%%% mode: latex
%%% TeX-master: "master"
%%% TeX-PDF-mode: t
%%% End:
