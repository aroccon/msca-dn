\section{Quality and Efficiency of the Implementation}

\subsection{Coherence and effectiveness of the work plan}
\note{including appropriateness of the allocation of tasks and resources
  (including awarding of the doctoral degrees for EID and EJD)}

\subsubsection{Work Packages description}
\note{please include table 3.1a}

\subsection*{Table 3.1a: Description of Work Packages}

% One for each WP
\note{%
\begin{itnwp}{MX -- MY}{Title}{Lead}
    \itnwppar{Objectives}{%
    }\\
    \hline    
    \itnwppar{Description of Work and Role of Specific Beneficiaries / Partner 
Organisations}{%
    }\\
    \itnwptask{Task name}{Task description}\\
    \hline
    \itnwppar{Description of Deliverables}{%
    }\\
\end{itnwp}
}

\subsubsection{List of major deliverables}
\note{please include table 3.1b, including the awarding of doctoral degrees, where applicable}

\note{%
    Type of deliverable:
    \begin{compactdesc}
    \item[R] Report
    \item[ADM] Administrative (website completion, recruitment completion, etc.)
    \item[PDE] dissemination and/or exploitation of results
    \item[OTHER] Other, including coordination.
    \end{compactdesc}
    Dissemination level:
    \begin{compactdesc}
    \item[PU] Public: fully open, e.g. web
    \item[CO] Confidential: restricted to consortium, other designated entities (as appropriate) and Commission services; Please consider that deliverables marked as "PU" will automatically be published on CORDIS once approved: the applicants should therefore consider the relevance of marking a deliverable as "PU"
    \item[CI] Classified: classified information as intended in Commission Decision 2001/844/EC
    \end{compactdesc}
}
\subsection*{Table 3.1b: Deliverables List}
\begin{itntable}{|>{\ra}p{15mm}|>{\ra}p{50mm}|>{\ra}p{15mm}|>{\ra}p{20mm}|>{\ra}p{15mm}|>{\ra}p{15mm}|>{\ra}p{15mm}|}
    \hline
    \rowcolor{olive!30}
    \multicolumn{7}{|l|}{\textbf{Scientific Deliverables}} \\
    \hline
    \rowcolor{olive!30}
    \textbf{Deliverable Number} &
    \textbf{Deliverable Title} &
    \textbf{WP No.} &
    \textbf{Lead Beneficiary Short Name} &
    \textbf{Type} &
    \textbf{Dissemination Level} &
    \textbf{Due Date} \\
    \hline
    & & & & & & \\
    \hline
    \rowcolor{olive!30}
    \multicolumn{7}{|l|}{\textbf{Management, Training, Recruitment and Dissemination Deliverables}} \\
    \hline
    \rowcolor{olive!30}
    \textbf{Deliverable Number} &
    \textbf{Deliverable Title} &
    \textbf{WP No.} &
    \textbf{Lead Beneficiary Short Name} &
    \textbf{Type} &
    \textbf{Dissemination Level} &
    \textbf{Due Date} \\
    \hline
    & & & & & & \\
    \hline
\end{itntable}


\subsubsection{List of major milestones}
\note{please include table 3.1c}

\subsection*{Table 3.1c: Milestones List}
\note{6 to 8 is OK}
\begin{itntable}{|>{\ra}p{15mm}|>{\ra}p{50mm}|>{\ra}p{15mm}|>{\ra}p{20mm}|>{\ra}p{15mm}|>{\ra}p{35mm}|}
    \hline
    \rowcolor{olive!30}
    \textbf{Number} &
    \textbf{Title} &
    \textbf{Related Work Packages} &
    \textbf{Lead Beneficiary} &
    \textbf{Due Data} &
    \textbf{Means of Verification} \\
    \hline
    & & & & & \\
    \hline
\end{itntable}

\subsubsection{Fellow's individual projects, including secondment plan}
\note{please include table 3.1d}
\subsection*{Table 3.1d: Individual Research Projects}
% Generic template; take a copy for each ESR
\note{%
\begin{itnrp}
    & & Y & M9 & 36 months & \\
    \hline
    \itnrppar{Project Title and Work Package(s) to which it is related}{%
    }\\ \hline
    \itnrppar{Objectives}{%
    }\\ \hline
    \itnrppar{Expected results}{%
    }\\ \hline
    \itnrppar{Planned secondment(s)}{%
    }\\
\end{itnrp}
}


\subsubsection{EID specific requirements}
\note{for EID proposals, an additional table should be completed in part B2}

The proposal is not an EID.

\subsection{Appropriateness of the management structures and procedures}
\note{including quality management and risk management (with a mandatory joint
  governing structure for EID and EJD)}

\subsubsection{Network organisation and management structure}
\note{including financial management strategy, strategy for dealing with
  scientific misconduct}

\subsubsection{Joint governing structure}
\note{mandatory for EID and EJD actions}

\subsubsection{Supervisory board}

\subsubsection{Recruitment strategy}

\subsubsection{Progress monitoring and evaluation of individual projects}

\subsubsection{Risk management}
\note{at consortium level (including table 3.2a)}

\subsubsection{Intellectual Property Rights (IPR)}

\subsubsection{Gender aspects}
\note{both at the level of recruitment and that of decision- making within the
  action}

\subsubsection{Data management plan}
\note{see page 26 above regarding the Open Access and Open Data under Horizon
  2020}

\subsection*{Table 3.2a: Implementation Risks}
\begin{itntable}{|>{\ra}p{10mm}|>{\ra}p{35mm}|>{\ra}p{15mm}|>{\ra}p{100mm}|}
    \hline
    \rowcolor{olive!30}
    \textbf{Risk No.} &
    \textbf{Description of Risk} &
    \textbf{WP Number} &
    \textbf{Proposed mitigation measures} \\
    \hline
    R1 & & WP1 & \\
    \hline
\end{itntable}


\subsection{Appropriateness of the infrastructure of the participating
  organisations}
\note{Explain the appropriateness of the infrastructure of each participating
  organisation, as outlined in Section 5 (Participating Organisations), in
  light of the tasks allocated to them in the action}

\subsection{Competences, experience and complementarity of the participating
  organisations and their commitment to the programme}

\subsubsection{Consortium composition and exploitation of participating
  organisations' complementarities}
\note{explain the compatibility and coherence between the tasks attributed to
  each beneficiary/partner organisation in the action, including in light of
  their experience}

\subsubsection{Commitment of beneficiaries and partner organisations to the
  programme}
\note{for partner organisations, please see also sections 5 and 7) i) Funding
  of non-associated third countries (if applicable): Only entities from EU
  Member States, from Horizon 2020 Associated Countries or from countries
  listed in General Annex A to the Work Programme are automatically eligible
  for EU funding. If one or more of the beneficiaries requesting EU funding is
  based in a country that is not automatically eligible for such funding, the
  application shall explain in terms of the objectives of the action why such
  funding would be essential. Only in exceptional cases will these
  organisations receive EU funding.49 The same applies for international
  organisations other than IEIO.  ii) Partner organisations: The role of
  partner organisations and their active contribution to the research and
  training activities should be described. A letter of commitment shall also be
  provided in section 7 and must follow the template (included within the PDF
  file, but outside the page limit)}

%%% Local Variables:
%%% mode: latex
%%% TeX-master: "master"
%%% TeX-PDF-mode: t
%%% End:
